Use of Octave (Mat\+L\+A\+B) in Civil Engineering Problems

To run this program, you need to install {\ttfamily G\+N\+U octave} on your computer. To know how to obtain it, visit\+: \href{https://www.gnu.org/software/octave/}{\tt https\+://www.\+gnu.\+org/software/octave/}

Once you have G\+N\+U Octave working, go to folder where code for an example is present. For example to run code correspond to example L01, go to folder Example/\+L01, ( for example on G\+N\+U/\+Linux you may do so by\+: {\ttfamily cd Example/\+L01} ). The code is in file {\ttfamily \hyperlink{namespacemain_af6e3698b7f50fc004eb759d7c447fdb3}{main.\+m}}, and data is in file {\ttfamily input.\+mat}. To store results in file {\ttfamily output.\+txt}, type command\+:

{\ttfamily octave \hyperlink{namespacemain_af6e3698b7f50fc004eb759d7c447fdb3}{main.\+m} $>$ output.\+txt}

or in following form, if we wish to produce pdf using La\+Te\+X from folder (tex, if available) using \hyperlink{run_8sh}{run.\+sh}

{\ttfamily octave \hyperlink{namespacemain_af6e3698b7f50fc004eb759d7c447fdb3}{main.\+m} $>$ output.\+tex}

View file output.\+txt in any text editor, or main.\+pdf (in tex folder) in pdf viewer.

Authors acknowledge used of function written by Sachin Shanbhag.

\href{http://sachinashanbhag.blogspot.in/2012/11/exporting-matrices-in-octavematlab-to.html}{\tt http\+://sachinashanbhag.\+blogspot.\+in/2012/11/exporting-\/matrices-\/in-\/octavematlab-\/to.\+html}

\href{https://docs.google.com/open?id=0Bww3OZktvGQucmxnd1FJNElCVGc}{\tt https\+://docs.\+google.\+com/open?id=0\+Bww3\+O\+Zktv\+G\+Qucmxnd1\+F\+J\+N\+El\+C\+V\+Gc}

\#\+Sage Use of Sage in Civil Engineering Problems

Requirement-\/\+:

\subsection*{1. Sagemath}

\subsection*{2. latex}

\subsection*{3. Django}