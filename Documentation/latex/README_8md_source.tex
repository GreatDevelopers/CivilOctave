\hypertarget{README_8md_source}{}\section{/home/amarjeet/projects/\+Civil\+Octave/\+R\+E\+A\+D\+M\+E.md}

\begin{DoxyCode}
00001 # CivilOctave
00002 Use of Octave (MatLAB) in Civil Engineering Problems
00003 
00004 To run this program, you need to install `GNU octave` on your computer.
00005 To know how to obtain it, visit: https://www.gnu.org/software/octave/
00006 
00007 Once you have GNU Octave working, go to folder where code for an example is
00008 present. For example to run code correspond to example L01, go to folder
00009 Example/L01, ( for example on GNU/Linux you may do so by: `cd Example/L01` ).
00010 The code is in file `main.m`, and data is in file `input.mat`. To store results
00011 in file `output.txt`, type command:
00012 
00013 `octave main.m > output.txt`
00014 
00015 or in following form, if we wish to produce pdf using LaTeX from folder
00016 (tex, if available) using run.sh
00017 
00018 `octave main.m > output.tex`
00019 
00020 View file output.txt in any text editor, or main.pdf (in tex folder) in pdf
00021 viewer.
00022 
00023 Authors acknowledge used of function written by Sachin Shanbhag.
00024 
00025 http://sachinashanbhag.blogspot.in/2012/11/exporting-matrices-in-octavematlab-to.html
00026 
00027 https://docs.google.com/open?id=0Bww3OZktvGQucmxnd1FJNElCVGc
00028 
00029 
00030 #Sage
00031 Use of Sage in Civil Engineering Problems 
00032 
00033 
00034 Requirement-:
00035 
00036 ## 1. Sagemath
00037 ## 2. latex 
00038 ## 3. Django
00039 
00040 
00041 
00042 
\end{DoxyCode}
